\documentclass[oneside, final, 14pt]{extreport}
\usepackage{amsthm, amsfonts, amsmath, amssymb, amscd}
\usepackage[T2A]{fontenc}
\usepackage[utf8]{inputenc}
\usepackage[english, russian]{babel}
\usepackage{graphicx}
\usepackage{indentfirst}
\usepackage{cite}
\usepackage{psfrag}

\IfFileExists{pscyr.sty}{\usepackage{pscyr}}{}    %

\usepackage[top=2cm,bottom=2cm,left=2.5cm,right=1cm]{geometry}

\linespread{1.3}

\newtheorem{theorem}{Теорема}
\newtheorem{lemma}{Лемма}
\newtheorem{definition}{Определение}

\begin{document}
\begin{titlepage}
\begin{center}
\includegraphics[width=8cm, height=4cm]{MSU}
\end{center}
\begin{center}
Московский государственный университет имени М.В. Ломоносова\\
\vspace{0.1 cm}
Факультет вычислительной математики и кибернетики\\
\vspace{0.1 cm}
Кафедра функционального анализа и его применений

\vspace{2cm}
{\large Анисимов Иван Александрович}\\
\vspace{1cm}

{\bf\Large Полнота системы синусов с комплексным параметром в пространстве функций интегрируемых по Лебегу}
\\ \vspace{2cm}
МАГИСТЕРСКАЯ ДИССЕРТАЦИЯ

\end{center}
 \vspace{2cm}
\begin{flushright}

{\bf Научный руководитель:}\\
Академик РАН, д-р. ф.-м.н.,\\
профессор Е.\,И. Моисеев

\end{flushright}

 \vspace{2cm}

\centerline {Москва, 2018}

\end{titlepage}

\tableofcontents

%-------------------------------
\newcommand\sincoeffs{\int\limits_0^{\pi}\sin(n - \frac{\beta}{2})xdx,
	\enspace n = 1, 2, \dots}
\newcommand\sinCNull{\int\limits_0^{\pi}f(x)\sin(n - \frac{\beta}{2})xdx = 0,
	\enspace n = 1, 2, \dots}
\newcommand\mainFunc{f_0(x) = \cos(\frac{x}{2})^{-\beta-1}\sin(\frac{x}{2})}
\newcommand\msin[1]{\sin(n - \frac{\beta}{2})#1}
\newcommand\mcos[1]{\cos(n - \frac{\beta}{2})#1}
\newcommand\mCoeff{n - \frac{\beta}{2}}
\newcommand\Real{\operatorname{Re}}
\newcommand\Imag{\operatorname{Im}}
\newcommand\pv[1]{\mathrel{\stackrel{\makebox[0pt]{\mbox{\normalfont\tiny п.в.}}}{#1}}}
%-------------------------------





\setcounter{chapter}{1}
\chapter*{Введение}
\addcontentsline{toc}{chapter}{Введение}

\section{История вопроса и актуальность темы}
Настоящая работа посвящена изучению свойств неортогональных систем синусов и косинусов в пространствах Лебега:
\begin{equation}
	\msin{x} \enspace n = 1, 2, \dots, \tag{I}\label{eq:I}
\end{equation}
\begin{equation}
	\mcos{x} \enspace n = 0, 1, 2, \dots \tag{II}\label{eq:II}
\end{equation}

Системы \eqref{eq:I}, \eqref{eq:II} в случае $\beta = 0$ хорошо изучены. Они подробно освещены в монографиях
А.~Зигмунда \cite{zigmund} и Н.~К.~Бари \cite{bari}. Случаю, когда $\beta \neq 0$, посвящены работы 
Н.~Винера, Р.~Пэли \cite{paley-wiener}, М.~И.~Кадеца \cite{kadec}, Е.~И.~Моисеева \cite{moiseev-dan}, 
Г.~Г.~Девдариани \cite{devdariani}, А.~М.~Седлецкого \cite{moiseev-sedleckiy}.

Научный интерес к изучению систем \eqref{eq:I} и \eqref{eq:II} не ограничивается лишь теоретическими вопросами. 
Рассматриваемые системы имеют важное приложение в краевых задачах для уравнения смешанного типа.
В 20-х годах XX века Ф.~Трикоми \cite{trikomi} положил начало исследования данных уравнений. 
Впервые задача нахождения решения уравнения смешанного типа была поставлена и исследована для уравнения
$$yu_{xx} + u_{yy} = 0,$$
которое сейчас называют уравнением Трикоми.

Исследования были продолжены в 30-е годы М.~Чибрарио и С.~Геллерстедтом. 
Обобщения результатов Ф.~Трикоми для уравнения
$$y^{2m+1}u_{xx} + u_{yy} = 0$$
принадлежат С.~Геллерстедту \cite{gellerstedt}.

С.~А.~Чаплыгин показал \cite{chaplygin}, что построение теории газовых струй тесно
связано с изучением уравнения
$$k(y)u_{xx} + u_{yy} = 0,$$
которое в настоящее время носит его имя. Коэффициент $k$ является
известной функцией переменного $y$, которая предполагается положительной
при $y > 0$ и отрицательной при $y < 0$. Случай $k(y) > 0$ соответствует
дозвуковому течению, а случай $k(y) < 0$ – сверхзвуковому течению газа.
При изучении околозвукового течения газа приходится иметь дело с
уравнением Чаплыгина в смешанных областях.

Началом нового этапа в развитии теории уравнений смешанного типа
явились работы Ф.~.И.~Франкля \cite{frankl-laval, frankl-eq}, в которых было показано, что задача
истечения сверхзвуковой струи из сосуда с плоскими стенками сводится к
задаче Трикоми для уравнения Чаплыгина. Также Ф.~И.~Франкль обнаружил важные приложения задачи Трикоми-Неймана в трансзвуковой
газодинамике, а также в теории сопел Лаваля. М.~А.~Лаврентьев предложил \cite{lavrentiev-bitsadze} более
простую модель уравнений смешанного типа
$$u_{xx} + sign(y) u_{yy} = 0$$
впоследствии названное уравнением Лаврентьева-Бицадзе. А.~В.~Бицадзе
осуществил подробное исследование этого уравнения.

И.~Н.~ Векуа указал \cite{vekua} на важность изучения задачи Дирихле для уравнений
смешанного типа в связи с задачами теории бесконечно малых изгибаний
поверхностей и безмоментной теории оболочек с кривизной переменного
знака. Возникла проблема нахождения областей, для которых задача Дирихле
окажется корректно поставленной, что явилось объектом дискуссии многих
авторов.

Работы Р.~Пэли и Н.~Винера \cite{paley-wiener}, Б.~Я.~ Левина \cite{levin} повысили интерес в исследовании 
систем экспонент:
$${A(t)e^{int}; B(t)e^{-ikt}}_{n \geqslant 0, k \geqslant 1}$$

Таким образом, практическая важность уравнений смешанного типа обуславливает 
актуальность изучения систем функций, с помощью которых решения выписываются в виде рядов.

\section{Основные понятия}

\begin{definition}
	\textit{Банаховым} пространством называется полное нормированное пространство	
	над полем вещественных или комплексных чисел.
\end{definition}

\begin{definition}
	\textit{Пространства $L_p(\chi, \mu)$} - это совокупность классов эквивалентных
	между собой функций, для которых $\int|f(x)|^pd\mu < \infty$, где $1 \leq p < \infty$. \\
	Пространство $L_p(\chi, \mu)$ является банаховым пространством относительно нормы 
	$$\|f\| = \left(\int|f(x)|^pd\mu\right)^{1/p}$$
\end{definition}

\begin{definition}
	Систему функций $\{\phi_k\}_{k = 1}^{\infty}$ будем называть полной в пространстве $L_1(0, \pi)$, если
	$\forall\thickspace\psi(x)\in L_1(0, \pi)$, т. ч. $\int\limits_0^\pi\phi(x)\psi(x)dx = 0$ следует, что $\psi(x) \pv{=} 0.$
\end{definition}




	
	
\section{Постановка задачи}


\chapter*{Результаты работы}
\addcontentsline{toc}{chapter}{Результаты работы}

\section{Система синусов в пространствах $L_p, p > 1$}

\section{Система синусов с действительным параметром $\beta$}

\section{Система синусов с коплексным параметром $\beta$}

Пусть \begin{equation}\sinCNull \tag{1} \label{eq:1}\end{equation}
	
	Для системы синусов известен следующий результат [4]:
	
	\medskip
	\textbf{Теорема.} \itshape Пусть $p \in (1, +\infty)$; тогда система синусов 
	$\sin(n - \frac{\beta}{2})\theta, n = 1, 2, \ldots$, 
	образует базис в $L_p(0, \pi)$ тогда и только тогда, 
	когда $\beta \in (-1/p, 2 - 1/p)$
	\normalfont

	В работе \cite{moi:gul} для системы 
	$$\sincoeffs$$ 
	был получен следующий результат:
	
	\begin{theorem}
		Пусть $f(x)$ интегрируема по Лебегу на $[0, \pi]$, выполнено условие
		\eqref{eq:1} и $\beta \in (0, 1)$, тода $f(x)$ почти всюду равняется нулю. 
		Если же  $\beta in (-1, 0)$, то существует функция $f_0(x)$, причем
		единственная, удовлетворяющая \eqref{eq:1} и равная $$\mainFunc.$$
	\end{theorem}
	
	Также из работы  \cite{dev:dis} известно:
	\begin{theorem}
		Пусть $p \in (1, \infty)$ , тогда система синусов $\sin(n - \beta / 2)\theta, \\
		n = 1, 2, \dots$ образует в пространстве $L_p[0, \pi]$ базис тогда и только тогда, 
		когда $\Real\beta \in (-1/p, 2 - 1/p)$.
	\end{theorem}
	
	
	Рассмотрим систему синусов $$\sincoeffs, \beta\in\mathbb{C}$$
	Тогда верно следующее:
	\begin{theorem}
		Пусть $f(x)$ интегрируема по Лебегу на $[0, \pi]$, выполнено условие
		\eqref{eq:1} и \thinspace $\Real\beta \in (0, 1)$, 
		тода $f(x)$ почти всюду равняется нулю. Если же \thinspace  
		$\Real\beta \in (-1, 0)$, то существует функция $f_0(x) \not\equiv 0$, 
		причем единственная, удовлетворяющая \eqref{eq:1} и равная $$\mainFunc.$$
	\end{theorem}
	
	
	\begin{proof}
		Пусть $f(x) \in L(0, \pi)$ и $$\sinCNull$$
		
		Воспользуемся равенством:
		$$
			\frac{\msin{x}}{n - \frac{\beta}{2}} = \int\limits_o^\pi\mcos{t}dt
		$$
		
		Имеем:
		
		$$
			 0 = \int\limits_0^\pi f(x)\msin{x}dx = \int\limits_0^\pi f(x) \int\limits_0^x\mcos{t}dtdx(\mCoeff)
		$$
		
		Меняем порядки интегрирования, используя теорему Фубини:
	
		$$
			\int\limits_0^\pi\mcos{t}F(t)dt = 0, n = 1, 2, \dots,
		$$
		где
		$$
			F(t) = \int\limits_t^\pi f(x) dx
		$$
		
		Заметим, что $F(x)$ является непрерывной функцией на $[0, \pi]$ в силу абсолютной непрерывности 
		интеграла Лебега \cite{kolmogorov}.
		
		В работе \cite{moiseev-dan} было доказано, что система косинусов
		$$
			\mcos{x}, n = 0, 1, 2, \dots
		$$
		
		образует базис в $L_p(0, \pi)$ при $p > 1$, причем биортогональная система имеет следующий вид
		$$
			h^c_k(x) = \frac{2C^k_\beta}{\pi(2\cos\frac{x}{2})^\beta}
			\Real\Bigl[F(-k; 1; \beta - k; -e^{ix}) - \frac{1}{2}\Bigr],
		$$
		
		гдe $F(a; b; c; z) - $ гипергеометрическая функция \cite{bateman}, следовательно $F(t)$ можео представить
		в виде ряда, сходящегося в $L_p(0, \pi)$
		$$
			F(t) = \sum\limits^\infty_{k = 0}B_k h^c_k(t), 
		$$
		
		где 
		$$
			B_k = \int\limits_0^\pi F(t) \mcos{t}dt,
		$$
		
		т.к. $B_k = 0, k \neq 0$, то отсюда получаем 
		$$
			F(t) = B_0 h^c_0(t)
		$$
		
		где 
		$$
			h^c_0 = \frac{\Gamma^2(1 - \frac{\beta}{2})}{2\pi \Gamma(1- \beta)(2\cos\frac{t}{2})^\beta}.
		$$
		
		Если $\Real\beta > 0$, то $h_0(t)$ имеет особенность в точке $t = \pi$, а функция $F(t)$ непрерывна на $(0, \pi)$,
		поэтому $B_0 = 0$ и $F(t) = 0$ на $[0, \pi]$.
		
		Если $F(t) = 0$, то 
		$$
			\int\limits_t^\pi f(x)dx = 0, t \in [0, \pi],
		$$
		
		отсюда следует, что $f(x)$ почти всюду равняется нулю на $[0, \pi]$.
		
		Рассмотрим случай $\Real\beta < 0$. Из \cite{moi:gul} известно, что:
		\begin{lemma}
			Если $\beta < 0, \beta\in\mathbb{R}$, то существует $f(x)$ интегрируемая, равная $h'_0(x)$ и 
			$$
				\sinCNull, \dots,
			$$
			
			гдe $h'_0(x) = const \times (-\beta\cos\frac{x}{2})\sin\frac{x}{2}$.
		\end{lemma}
		
		Так как функция $h'_0(x)$ является непрерывной, мы можем аналитически продолжить ее на область 
		$\Real\beta < 0$. Покажем, что выполняется условие \eqref{eq:1}:
		
		 Рассмотрим следующий интеграл:
		$$
			\int\limits_0^\pi\cos(\frac{x}{2})^{-\beta - 1}\sin(\frac{x}{2})\sin(n - \frac{\beta}{2})xdx = 
		$$
		
		\begin{equation}
			2\int\limits_0^\frac{\pi}{2}\cos(t)^{-\beta - 1}\sin(t)\sin(2n - \beta)tdt 
			= \int\limits_0^\frac{\pi}{2}cos(t)^{-\beta-1}\bigl(\cos(at) - \cos(bt)\bigr)dt,	\tag{2} \label{eq:2}
		\end{equation}
		где $a =1 + \beta - 2n, b = 1 - \beta + 2n.$
		
		
		
		\bigskip
		
		Воспользуемся формулой из \cite{bateman}:
		$$
			\int\limits_0^\frac{\pi}{2}(\cos(t))^\alpha \cos(\gamma t)dt = \frac{\pi}{2^{\alpha + 1}}
			\frac{\Gamma(1 + \alpha)}{\Gamma(1 + \frac{\alpha + \gamma}{2})\Gamma(1 + \frac{\alpha - \gamma}{2})},
			\enspace\Real\alpha > -1
		$$
		
		Таким образом, из \eqref{eq:2} получаем:
		$$
			\int\limits_0^\pi\cos(\frac{x}{2})^{-\beta - 1}\sin(\frac{x}{2})\sin(n - \frac{\beta}{2})xdx = 
		$$
		
		$$
			\frac{\pi\Gamma(-\beta)}{2^{-\beta}}\Bigl[\frac{1}{\Gamma(1 - n)\Gamma(-\beta + n)} - 
			\frac{1}{\Gamma(1 -\beta + n)\Gamma(- n)}\Bigr]
		$$
		
		Единственность $f(x)$ следует из единственности биортогональной системы \cite{moiseev-dan}.
		
		Таким образом, теорема доказана.
		
	\end{proof}




	
\chapter*{Заключение}
\addcontentsline{toc}{chapter}{Заключение}
	
\begin{thebibliography}{30}
	\addcontentsline{toc}{chapter}{Литература}

	\bibitem{bateman}
	{\it Bateman H., Erd{\'e}lyi A.} Higher transcendental functions — 
	McGraw-Hill — 1953.
	
	\bibitem{gellerstedt}
	{\it Gellerstedt S.} Sur une equation lineaire aux derivees partielles de type mixte. — 
	Ark Mat Astron Fys A. — 1937 —25(29), 1–23. 

	\bibitem{kadec}
	{\it Kadec M.I.} The exact value of the PaleyWiener constant, 
	Sov. Math. Dokl. 5, 1964, 559561

	\bibitem{paley-wiener}
	{\it Paley R.E.A.C., Wiener N.} Fourier transforms in the complex domain. 
	New York: American Mathematical Society. — 1934. — Vol. 19. — P. x+ 184.
	
	\bibitem{zigmund}
	{\it Zygmund, A.} Trigonometric Series. 
		Warszawa: Z Subwencji Funduszu Kultury Narodowej. — 1935. — 331 p.
	
	\bibitem{bari}
	{\it Бари Н. К.} Тригонометрические ряды. —
	М.: Физматлит, 1961. — 936 с.
	
	\bibitem{bitsadze}
	{\it Бицадзе А. В. } О некоторых задачах для уравнений смешааного типа. —
	Докл. АН СССР. 1950. Т. 70, № 4. С. 561–564.
	
	\bibitem{vekua}
	{\it Векуа И.Н.} Некоторые общие методы построения различных вариантов теории оболочек. — 
	Москва: Изд-во «Наука». — 1982. — 288 С.
	
	\bibitem{devdariani}
	{\it Девдариани Г. Г.} Базисность некоторых специальных систем собственных функций несамосоиряжённых дифференциальных операторов. —
	Автореф. дисс. . канд. физ.-мат. наук. Москва: МГУ, 1986.
	
	\bibitem{kolmogorov}
	{\it Колмогоров А.Н., Фомин С.В.} Элементы теории функций и функционального анализа. —
	Наука, 1989.
	
	\bibitem{lavrentiev-bitsadze}
	{\it Лаврентьев М.А., Бицадзе А.В.} К проблеме уравнений смешанного типа — 
	Докл. АН СССР. — 1950. — Т. 79. — №. 3. — С. 373–376.

	\bibitem{levin}
	{\it Левин Б.Я.} Распределение корней целых функций. — 1956. — 632 С.
	
	\bibitem{moiseev-dan}
	{\it Моисеев Е. И.} О базисности систем косинусов и синусов. —
	Докл. АН СССР. 1984. Т. 275, № 4. С. 794–798.
	
	\bibitem{moiseev-sedleckiy}
	{\it Моисеев Е.И., Прудников А.П., Седлецкий А.М.} Базисность и полнота некоторых тригонометрических систем элементарных функций. —
	ВЦ им. А.А. Дородницына РАН. Москва. — 2004. — 146 C.
	
	\bibitem{trikomi}
	{\it Трикоми Ф.} О линейных уравнениях смешанного типа (пер. с итал.) —
	М.–Л.: Изд-во «Гостехиздат». — 1947. — 192 С.
	
	\bibitem{frankl-laval}
	{\it Франкль Ф. И.} К теории сопел Лаваля —
	Изв. АН СССР. Сер.: Математика, 1945, Т. 9, № 5, С. 387-422.
	
	\bibitem{frankl-eq}
	{\it Франкль Ф. И.} К теории уравнения $yZ_{xx} + Z_{yy} = 0$ —
	Изв. АН СССР, Сер.: Математика, 1946, Т. 10, № 2, С. 135-166.
	
	\bibitem{chaplygin}
	{\it Чаплыгин С. А.}  О газовых струях. — М., 1902. — 121 с.
	
\end{thebibliography}
\end{document} 